\documentclass[catalog.tex]{subfiles}

% do not write anything in the preamble

\begin{document}

\def\pbname{Gaussian Blur} %change this, do not use any number, just the name

\section{\pbname} 

% only for overview, so short description (no more than 1-2 lines)
\begin{overview}
\item [Algorithm:] Gaussian Blur~(algo.~\ref{alg:\currfilebase_b}) 
	% -	must match the label of the algorithm 
	% - when writing more than one algo use alg:\currfilebase_a, alg:\currfilebase_b, etc.
\item [Input:] An image
\item [Complexity:]$\mathcal{O}(w_kw_ih_i)+\mathcal{O}(h_kw_ih_i)$, depending on the size of filter kernel and the image
\item [Data structure compatibility:] Image
\item [Common applications:] Image processing, edge detection, computer vision
\end{overview}


\begin{problem}{\pbname}
	Gaussian blur is a type of image blurting filter that uses a Gaussian function to reduces details and noise in images.
\end{problem}


\subsection*{Description}
\subsubsection{Algorithm Description}
\par Gaussian blur, or Gaussian smoothing, uses a Gaussian function to calculating the transformation of pixel in an image. Gaussian blur is widely used to graphics or image processing software. Gaussian blur serves as a pre-processing stage for lots of computer vision algorithms in an attempt to enhance image structures. 
\par The formula of an 1-D Gaussian function is expressed as
$$
	G(x)=\frac{1}{\sqrt{2 \pi \sigma^{2}}} e^{-\frac{x^{2}}{2 \sigma^{2}}}
$$
A 2-D Gaussian function is the product of two 1-D Gaussian function, expressed as~\cite{d_i_p}
$$
G(x, y)=\frac{1}{2 \pi \sigma^{2}} e^{-\frac{x^{2}+y^{2}}{2 \sigma^{2}}}
$$
where $x$ is the distance from the point (or the origin) in horizon. $y$ is the distance from the point (or origin) in vertex. $\sigma$ is the standard deviation of the Gaussian distribution.
The idea of Gaussian blurring is implemented by convolution. Such convolution is performed by convolving with 1-D Gaussian function first in the horizonal, then the vertical direction. 
\par When calculating a discrete approximation of the Gaussian function, pixels that have a distance of more than three standard deviations can be consider effectively zero~\cite{uk} since they have little influence. Therefore, contribution from pixels farther than $3\sigma$ can be ignored. A filter kernel of a matrix with $\lceil 6 \sigma\rceil \times\lceil 6 \sigma\rceil$ is enough. 
\par The overall running time is determined by size of the filter kernel and the image itself
$$
	T=\mathcal{O}(w_kw_ih_i)+\mathcal{O}(h_kw_ih_i)
$$
where $w_k,h_k$ is the width and height of the filter kernel, $w_i,h_i$ is the width and height of the image.
% add comment in the pseudocode: \cmt{comment}
% define a function name: \SetKwFunction{shortname}{Name of the function}
% use the defined function: \shortname{$variables$}
% use the keyword ``function'': \Fn{function name}, e.g. \Fn{\shortname{$var$}}
\newpage
\subsubsection{Implementation}
The filter kernel is generated by 2-D Gaussian function. The size $N$ of the filter kernel $N \times N$ has to be odd. Take the center element as the origin point, horizontal direction and right as $x-axis$ and positive direction, vertical direction and up as $y-axis$ and positive direction. All the value is calculated through this coordinate system. The sum of all the value in a filter kernel should be exactly $1$.[~\cite{MachineVision}] In this algorithm, the filter kernel is a matrix of $\lceil 6 \sigma\rceil \times\lceil 6 \sigma\rceil$.
\begin{Algorithm}[$getFilterKernel(\sigma)$\label{alg:\currfilebase_a}][ht]
	% -	must match the reference in the overview
	% - when writing more than one algo use alg:\currfilebase_a, alg:\currfilebase_b, etc.
	%\SetKwFunction{myfunction}{MyFunction}	
	\Input{standard deviation $\sigma$}
	\Output{a $\lceil 6 \sigma\rceil \times\lceil 6 \sigma\rceil$ matrix $A$}
	%	\Fn{\myfunction{$a,b$}}{
	%	}
	\BlankLine
	\For{$i=-\lceil 6 \sigma \rceil /2$ \textbf{to} $\lceil 6 \sigma \rceil/2$}{
		\For{$j=\lceil 6 \sigma \rceil /2$ \textbf{to} $\lceil 6 \sigma \rceil/2$}{
			$A[i][j]$ $\gets$ $\frac{1}{2 \pi \sigma^{2}} e^{-\frac{i^{2}+j^{2}}{2 \sigma^{2}}}$
		}
	}
	$sum$ $\gets$ the sum of all elements in $A$ \\
	$A$ $\gets$ $A./sum$ \\
	\Ret $A$

\end{Algorithm}

\newpage
\begin{Algorithm}[Gaussian Blur\label{alg:\currfilebase_b}]
	% -	must match the reference in the overview
	% - when writing more than one algo use alg:\currfilebase_a, alg:\currfilebase_b, etc.
	%\SetKwFunction{myfunction}{MyFunction}	
	\Input{An image $pix$, standard deviation $\sigma$}
	\Output{Blurred image $pix$}
	\BlankLine
	GaussM = $\textbf{getFilterKernel}(\sigma)$ \\
	%	\Fn{\myfunction{$a,b$}}{
	%	}
	radius $\gets$ $\lceil 6 \sigma \rceil /2$ \\
	\cmt{Processing on the horizonal direction} \\
	\For{every row of pixels}{
		\cmt{For the $\textbf{i}th$ row} \\
		$GausSum$,$rSum$,$gSum$,$bSum$ $\gets$ $0$ \\
		\For{ervey column of pixels}{
			\cmt{For the $\textbf{j}th$ column} \\
			\For{k=-randius \textbf{to} radius}{
				cur $\gets$ $j+k$ \\
				\If{$0 \leq cur \leq \text{width of the image}$}{
					$r$,$g$,$b$ $\gets$ rgb value of $pix[i][cur]$ \\
					$rSum$ $\gets$ $rSum+r$ \\
					$gSum$ $\gets$ $gSum+g$ \\
					$bSum$ $\gets$ $bSum+b$ \\
					$GausSum$ $\gets$ $GaussM[k+radius]$ \\
				}
			}
			\cmt{The processed rgb value} \\
			$r$ $\gets$ $rSum/GausSum$ \\
			$g$ $\gets$ $gSum/GausSum$ \\
			$b$ $\gets$ $bSum/GausSum$ \\
			$pix[i][j]$ $\gets$ $r << 16 \ |\  g \ << 8 \ | \ b\  | \ 0xff000000$ 
		}
	}

	\cmt{Processing on the vertical direction} \\
	\For{every column of pixels}{
		\cmt{For the $\textbf{i}th$ column} \\
		$GausSum$,$rSum$,$gSum$,$bSum$ $\gets$ $0$ \\
		\For{ervey row of pixels}{
			\cmt{For the $\textbf{j}th$ row} \\
			\For{k=-randius \textbf{to} radius}{
				cur $\gets$ $j+k$ \\
				\If{$0 \leq cur \leq \text{width of the image}$}{
					$r$,$g$,$b$ $\gets$ rgb value of $pix[i][cur]$ \\
					$rSum$ $\gets$ $rSum+r$ \\
					$gSum$ $\gets$ $gSum+g$ \\
					$bSum$ $\gets$ $bSum+b$ \\
					$GausSum$ $\gets$ $GaussM[k+radius]$ \\
				}
			}
			\cmt{The processed rgb value} \\
			$r$ $\gets$ $rSum/GausSum$ \\
			$g$ $\gets$ $gSum/GausSum$ \\
			$b$ $\gets$ $bSum/GausSum$ \\
			$pix[i][j]$ $\gets$ $r << 16 \ |\  g \ << 8 \ | \ b\  | \ 0xff000000$ 
		}
	}



	\Ret

\end{Algorithm}

\newpage
% include references where to find information on the given problem using latex bibliography
% insert references in the text (\cite{}) and write bibliography file in problem-nb.bib (replace nb with the problem number)
% prefer books, research articles, or internet sources that are likely to remain available over time
% as much as possible offer several options, including at least one which provide a detailed study of the problem
% if available include links to programs/code solving the problem
% wikipedia is NOT acceptable as a unique reference
\singlespacing
\printbibliography[title={References.},resetnumbers=true,heading=subbibliography]

\end{document}
