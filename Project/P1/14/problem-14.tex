\documentclass[catalog.tex]{subfiles}

% do not write anything in the preamble

\begin{document}

\def\pbname{Sorting (Merge sort, quick sort, heap sort)} %change this, do not use any number, just the name

\section{\pbname} 

% only for overview, so short description (no more than 1-2 lines)
\begin{overview}
\item [Algorithm:] Sorting (Merge sort~(algo.~\ref{alg:\currfilebase_a}), quick sort~(algo.~\ref{alg:\currfilebase_c}), heap sort~(algo.~\ref{alg:\currfilebase_d})) 
	% -	must match the label of the algorithm 
	% - when writing more than one algo use alg:\currfilebase_a, alg:\currfilebase_b, etc.
\item [Input:] An unsorted array
\item [Complexity:] $\mathcal{O}(n\log n)$ for all of three
\item [Data structure compatibility:] Heap for heap sort, array for merge sort and quick sort
\item [Common applications:] Foundation for algorithms (i.e. binary search requires sorted sequence), Data processing in excels and tables
\end{overview}


\begin{problem}{\pbname}
	A sorting algorithm places elements of an sequence in a certain order.
\end{problem}


\subsection*{Description}
\subsubsection{Problem Description}
Sorting is a fundamental problem in algorithms. Quite a few search or graph algorithm requires a sorted sequence as input. When studying sorting algorithm, the concept of an \textbf{in place} algorithm is needed. An \textbf{in place} algorithm is an algorithm that requires $\mathcal{O}(1)$ extra space.
\subsubsection{Algorithm Description}
A sorting probem is defined as followed:~\cite{intro3rd} \\
\textbf{Input: }A sequence of $n$ numbers $(a_1,a_2,\cdots,a_n)$. \\
\textbf{Output: }A permutation $(a_1^{'},a_2^{'},\cdots,a_n^{'})$ of the input sequence such that $a_1^{'} \leq a_2^{'} \leq \cdots \leq a_n^{'}$\\
\subsubsection{Merge sort}
Merge sort is a divide-and-conquer algorithm. It keeps merging two sorted sequences into a new sorted sequences. Let $T(i)$ be the time of sorting $i$ elements. Since merging two sequence together takes $\mathcal{O}(N)$ time, scanning at most all the elements.
$$
	T(N)=2T(N/2)+N;
$$
According to Master Theorem, the complexity of merge sort can be expressed as
$$
	T(N)=\mathcal{O}(N \log N)
$$
% add comment in the pseudocode: \cmt{comment}
% define a function name: \SetKwFunction{shortname}{Name of the function}
% use the defined function: \shortname{$variables$}
% use the keyword ``function'': \Fn{function name}, e.g. \Fn{\shortname{$var$}}
\begin{Algorithm}[$mergeSort(A,l,r)$\label{alg:\currfilebase_a}]
	% -	must match the reference in the overview
	% - when writing more than one algo use alg:\currfilebase_a, alg:\currfilebase_b, etc.
	%\SetKwFunction{myfunction}{MyFunction}	
	\Input{A subsequence of $A$: $A[l:r+1]$}
	\Output{A sorted sequence $C$ with $r-l+1$ numbers}
	%	\Fn{\myfunction{$a,b$}}{
	%	}
	\BlankLine
	\If{$l \geq r$}{
		\Ret
	}
	$mid$ $\gets$ $(l+r)/2$ \\
	$mergeSort(A,l,mid)$ \\
	$mergeSort(A,mid+1,r)$ \\
	\cmt{Merge the left sequence and right sequence togther.} \\
	$i$ $=$ $j$ $=$ $k$ $=$ $0$ \\
	\While{$i < (mid-l+1)$ \textbf{and} $j < (r-mid)$}{
		\If{$A[i] \leq B[j]$}{
			$C[k++]=A[j++]$
		}
		\Else{
			$C[k++]=B[j++]$
		}
	}
	\If{$i = (mid-l+1)$}{
		Append $B[j:r+1]$ to $C$.
	} 
	\Else{
		Append $A[i:mid+1]$ to $C$.
	}
	\Ret $C$

\end{Algorithm}

Since merge sort takes $\mathcal{O}(N)$ extra place to store the stored sequence, it is not an \textbf{in place} algorithm.
Another implementation of merge sort takes $\mathcal{O}(1)$ extra place, which is \textbf{in place}.
\newpage
\begin{Algorithm}[$inPlaceMergeSort(A,l,r)$\label{alg:\currfilebase_b}]
	% -	must match the reference in the overview
	% - when writing more than one algo use alg:\currfilebase_a, alg:\currfilebase_b, etc.
	%\SetKwFunction{myfunction}{MyFunction}	
	\Input{A subsequence of $A$: $A[l:r+1]$}
	\Output{Sorted sequence $A$}
	%	\Fn{\myfunction{$a,b$}}{
	%	}
	\BlankLine
	\BlankLine
	\If{l \textbf{$\geq$} r}{
		\textbf{return}
	}
	MergeSort($A$,$l$,$n/2$) \\
	MergeSort($A$,$n/2+1$,$r$) \\
	$i$ $\gets $ $l$ \\
	$j$ $\gets$ $n/2+1$ \\
	$tmp$ $\gets$ $j$ \\
	\While{$i < r$}{
	\While{$a[i] < a[j]$}{i++}
	\While{$a[i] > a[j]$}{j++}
	swap \textbf{a[i:index]} with \textbf{a[index:j]} \\
	$i$ $\gets$ $i+j-tmp$}
	\Ret $A$
\end{Algorithm}
\subsubsection{Quick sort}
Quick sort is a divide-and-conquer algorithm. It works by placing one of the elements into its right place every time. This element is called a \textbf{pivotat}. The running time varies between $\mathcal{O}(N)$ and $\mathcal{O}(N^2)$~\cite{ve477} depending on the pivotat selected. Let $T(i)$ be the time of sorting $i$ elements. Partition the sequence around $pivot$ takes $\mathcal{O}(N)$ time, as the worst case is to scan all the elements.
Therefore,
$$
	T(N)=2T(N/2)+cN
$$
According to Master Theorem, the average running time, thus the time complexity of quickSort sort can be expressed as
$$
	T(N)=\mathcal{O}(N \log N)
$$
Quick sort is an \textbf{in place} algorithm since $\mathcal{O}(1)$ extra place is used.

\begin{Algorithm}[$quickSort(A,l,r)$\label{alg:\currfilebase_c}]
	% -	must match the reference in the overview
	% - when writing more than one algo use alg:\currfilebase_a, alg:\currfilebase_b, etc.
	%\SetKwFunction{myfunction}{MyFunction}	
	\Input{A subsequence of $A$: $A[l:r+1]$}
	\Output{Sorted sequence $A$ }
	%	\Fn{\myfunction{$a,b$}}{
	%	}
	\BlankLine
	\If{$l \geq r$}{
		\Ret
	}
	$pivotIndex$ $\gets$ a random number between $l$ and $r$ \\
	swap $A[pivotIndex]$ and $A[0]$ \\
	$pivot$ $\gets$ $A[0]$ \\
	$i$ $\gets$ $1$ \\
	$j$ $\gets$ $N-1$ \\
	\cmt{Partition the sequence around pivot}
	\While{$i<j$}{
		\While{$A[i] < pivot$}{
			$i++$
		}
		\While{$A[j] \geq pivot$}{
			$j--$
		}
		\If{$i<j$}{
			swap $A[i]$ with $A[j]$ \\
		}
	}
	swap $A[0]$ and $A[j]$ \\
	$quickSort(A,l,j-1)$ \\
	$quickSort(A,j+1,r)$ \\
	\Ret $A$

\end{Algorithm}

\subsubsection{Heap Sort}
Heap sort is based on data structure \textbf{heap}. A heap is based on a complete binary tree, has the property that for any node other than the root~\cite{intro3rd}
$$
	A[\lfloor i / 2\rfloor] \leq A[i]
$$
Procedures that are used in heap sort shows in the following.~\cite{ve281}
\begin{enumerate}
	\item The $maintain$ procedure is to maintain the property of a heap. 
	\item The $buildHeap$ procedure is to produces a heap from an unsorted array.
\end{enumerate}
Since the height of a complete bianry tree of $N$ elements is $\Theta(\log N)$, $maintain$ takes $\mathcal{O}(\log N)$ running time. For one complete heap sort, there are $\mathcal{O}(N)$ loops. Therefore, the time complexity of heap sort can be expressed as
$$
	T(N) = \mathcal{O}(N) \times \mathcal{O}(\log N) = \mathcal{O}(N\log N)
$$

\begin{Algorithm}[$heapSort(A)$\label{alg:\currfilebase_d}]
	% -	must match the reference in the overview
	% - when writing more than one algo use alg:\currfilebase_a, alg:\currfilebase_b, etc.
	\SetKwFunction{myfunction}{maintain}
	\SetKwFunction{anotherfunction}{buildHeap}
	\Input{An unsorted sequence $A$}
	\Output{None}
	%	\Fn{\myfunction{$a,b$}}{
	%	}
	\Fn{\myfunction{$A,i$}}{
		$l \gets 2i$ \\
		$r \gets 2i+1$ \\
		\If{$l \leq A.heap\_size$ \textbf{and} $A[l] > A[i]$}{
			$largest \gets l$
		}
		\Else{
			$largest \gets i$
		}
		\If{$r \leq A.heap\_size$ \textbf{and} $A[r] > A[largest]$}{
			$largest \gets r$
		}
		\If{$largest \neq i$}{
			swap $A[i]$ with $A[largest]$ \\
			$maintain(A,largest)$
		}
	}
	\Fn{\anotherfunction{$A,i$}}{
		$A.heap\_size$ $\gets$ $A.length$ \\
		\For{$i=\lfloor A \cdot \operatorname{length} / 2\rfloor$ \textbf{downto} $1$}{
			\myfunction{$A$,$i$}
		}
	}
	\BlankLine
	
	\anotherfunction($A$) \\
	\For{$i=A.length$ \textbf{downto} $2$}{
			swap $A[1]$ with $A[i]$ \\
			$A.heap\_size--$\\
			\myfunction{$A$,$i$}
		}
	\Ret


\end{Algorithm}
\newpage




% include references where to find information on the given problem using latex bibliography
% insert references in the text (\cite{}) and write bibliography file in problem-nb.bib (replace nb with the problem number)
% prefer books, research articles, or internet sources that are likely to remain available over time
% as much as possible offer several options, including at least one which provide a detailed study of the problem
% if available include links to programs/code solving the problem
% wikipedia is NOT acceptable as a unique reference
\singlespacing
\printbibliography[title={References.},resetnumbers=true,heading=subbibliography]

\end{document}
