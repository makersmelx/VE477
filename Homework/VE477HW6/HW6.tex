\documentclass[12pt,a4paper]{article}
%\usepackage{ctex}
\PassOptionsToPackage{unicode}{hyperref}
\PassOptionsToPackage{hyphens}{url}
\usepackage{amsmath,amscd,amsbsy,amssymb,latexsym,url,bm,amsthm}
\usepackage{epsfig,graphicx,subfigure}
\usepackage{enumitem,balance}
\usepackage{wrapfig}
\usepackage{mathrsfs,euscript}
\usepackage[usenames]{xcolor}
\usepackage[colorlinks,linkcolor = blue]{hyperref}
\usepackage[vlined,ruled,commentsnumbered,linesnumbered]{algorithm2e}
\usepackage{listings}
\usepackage[utf8]{inputenc}
\usepackage{algorithmicx}
\usepackage{algpseudocode}
\usepackage{amsmath}

\newtheorem{theorem}{Theorem}
\newtheorem{lemma}[theorem]{Lemma}
\newtheorem{proposition}[theorem]{Proposition}
\newtheorem{corollary}[theorem]{Corollary}
\newtheorem{exercise}{Exercise}
\newtheorem*{solution}{Solution}
\newtheorem{definition}{Definition}
\theoremstyle{definition}


%\numberwithin{equation}{section}
%\numberwithin{figure}{section}

\renewcommand{\thefootnote}{\fnsymbol{footnote}}

\newcommand{\postscript}[2]
 {\setlength{\epsfxsize}{#2\hsize}
  \centerline{\epsfbox{#1}}}

\renewcommand{\baselinestretch}{1.0}

\setlength{\oddsidemargin}{-0.365in}
\setlength{\evensidemargin}{-0.365in}
\setlength{\topmargin}{-0.3in}
\setlength{\headheight}{0in}
\setlength{\headsep}{0in}
\setlength{\textheight}{10.1in}
\setlength{\textwidth}{7in}
\makeatletter \renewenvironment{proof}[1][Proof] {\par\pushQED{\qed}\normalfont\topsep6\p@\@plus6\p@\relax\trivlist\item[\hskip\labelsep\bfseries#1\@addpunct{.}]\ignorespaces}{\popQED\endtrivlist\@endpefalse} \makeatother
\makeatletter
\renewenvironment{solution}[1][Solution] {\par\pushQED{\qed}\normalfont\topsep6\p@\@plus6\p@\relax\trivlist\item[\hskip\labelsep\bfseries#1\@addpunct{.}]\ignorespaces}{\popQED\endtrivlist\@endpefalse} \makeatother

\SetKwInOut{Input}{Input}
\SetKwInOut{Output}{Output}
\SetKwProg{Fn}{Function}{\string:}{end}

\usepackage{color}
\definecolor{codegreen}{rgb}{0,0.6,0}
\definecolor{codegray}{rgb}{0.5,0.5,0.5}
\definecolor{codepurple}{rgb}{0.58,0,0.82}
\definecolor{backcolour}{rgb}{0.95,0.95,0.92}
 
\lstdefinestyle{mystyle}{  
    commentstyle=\color{codegreen},
    keywordstyle=\color{blue},
    numberstyle=\tiny\color{codegray},
    stringstyle=\color{codepurple},
    basicstyle=\footnotesize,
    breakatwhitespace=false,         
    breaklines=true,                 
    captionpos=b,                    
    keepspaces=true,                 
    numbers=left,                    
    numbersep=5pt,                  
    showspaces=false,                
    showstringspaces=false,
    showtabs=false,                  
    tabsize=2,
    frame=shadowbox
}
\lstset{style=mystyle}

\title{VE477 HW 6}
\author{Wu Jiayao 517370910257}

\begin{document}
\maketitle
\section{Perfect matching in a bipartite graph}
\subsection{Proof}
Suppose that matrix A has m rows and n columns. m = n. Denote $M^1_{i,j}$ the minor $M_{i,j}$ of A, $M^{k}_{i,j}$ as the minor $M_{i,j}$ of $M^{k-1}$
\begin{enumerate}
    \item If the determinant is identically zero. Suppose there is a perfect matching, then for every row in A, there is only one element that has a value 1, denote as $a_{i_x,j_x}$ for $x^{th}$ row. The determinant of A is expressed as
    \begin{align*}
        det(A) &= (-1)^{i_1+j_1} \cdot M^{1}_{i_1,j_1} + 0 \times (n-1) \\
        &= (-1)^{i_1+j_1+i_2+j_2} \cdot M^{2}_{i_2}{j_2} + 0 \times (n-2)\\
        &= \cdots \\
        &= (-1)^{i_1+j_1+\cdots+i_{m-1}+j_{m-1}} \cdot M^{m-1}_{i_{m-1},j_{m-1}} \\
        &= (-1)^{x} \neq 0
    \end{align*}
    Since $M^{m-1}_{i_{m-1},j_{m-1}}$ is 1, as it represents in the perfect matching the match of the $(m-1)^{th}$ vertex in L, such leads to a contradiction. Hence there is no perfect matching.
    \item If G has no perfect matching, there is at least one vertex in L that has no vertext in R to match, which means that there exists at least one row in A that is all 0. The determinant is identically zero in this condition.
\end{enumerate}
\subsection{Algorithm}
Use the maximum matching algorithm described in the lecture. \\
\begin{algorithm}[H]
    \Input{a bipartite graph $G \langle V,E \rangle$, $V = L \cup R$}
    \Output{Whether it has a perfect matching}
    Add $s$ and $t$ \\
    Connect $s$ with all vertice in $L$ with capacity 1 \\
    Connect $t$ with all vertice in $R$ with capacity 1 \\
    \For{each edge in $G$}{
        Its capacity is 1 
    }
    h = Ford - Fulkson(G) \\
    \If{$h =|R|$}{
        \Return true
    }
    \Return false
\end{algorithm}
\subsection{Complexity and error probability}
The complexity is $\mathcal{O}(|V||E|)$. Since the existence of perfect matching matches the maximum of flows, the algorithm is always correct, with 0\% of error probability.
\subsection{Usefulness}
It is correct. It has a resonable time complexity.
\section{Critical Thinking}
\subsection{The middle node}
\begin{algorithm}[H]
    \Input{A singly linked list}
    \Output{The middle node}
    $slow$ $=$ the first node of the singly linked list \\
    $fast$ $=$ $slow \rightarrow next$ \\
    \While{$fast$ is not the last node of the singly linked list}{
        $slow$ $=$ $slow \rightarrow next$ \\
        $fast$ $=$ $fast \rightarrow next \rightarrow next$ \\
    }
    \Return $slow$
\end{algorithm}
\subsection{The cycle}
\begin{algorithm}[H]
    \Input{A singly linked list}
    \Output{If there is a cycle}
    $slow$ $=$ the first node of the singly linked list \\
    $fast$ $=$ $slow \rightarrow next$ \\
    \While{$fast$ is not the last node of the singly linked list}{
        \If{$fast\ ==\ slow$}{
            \Return True
        }
        $slow$ $=$ $slow \rightarrow next$ \\
        $fast$ $=$ $fast \rightarrow next \rightarrow next$ \\
    }
    \Return False
\end{algorithm}
The complexity is $\mathcal{O}(n)$, where $n$ is the length of the singly linked node.
\section{The coupon collector desillusion}
\subsection{How many boxes}
At least n.
\subsection{Not finished}
\subsection{Expectation}
$$
E[X]=\sum_{i=1}^{n} \frac{n}{i}=n \sum_{i=1}^{n} \frac{1}{i} \approx n \int_{1}^{n} \frac{1}{x} d x=n \log n
$$
It can be concluded that $E[X]=\Theta(n\log n)$
\subsection{Coupon collector}
The more boxes you have already owned, the more boxes you expect to buy (the bigger expectation) to collect a new boxes.
% \section*{Reference}
% {\noindent [1] \url{cs.nthu.edu.tw/~wkhon/toc07-lectures/lecture21.pdf}}
\end{document}