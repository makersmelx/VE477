\documentclass[12pt,a4paper]{article}
%\usepackage{ctex}
\usepackage{amsmath,amscd,amsbsy,amssymb,latexsym,url,bm,amsthm}
\usepackage{epsfig,graphicx,subfigure}
\usepackage{enumitem,balance}
\usepackage{wrapfig}
\usepackage{mathrsfs,euscript}
\usepackage[usenames]{xcolor}
\usepackage{hyperref}
\usepackage[vlined,ruled,commentsnumbered,linesnumbered]{algorithm2e}

\newtheorem{theorem}{Theorem}
\newtheorem{lemma}[theorem]{Lemma}
\newtheorem{proposition}[theorem]{Proposition}
\newtheorem{corollary}[theorem]{Corollary}
\newtheorem{exercise}{Exercise}
\newtheorem*{solution}{Solution}
\newtheorem{definition}{Definition}
\theoremstyle{definition}


%\numberwithin{equation}{section}
%\numberwithin{figure}{section}

\renewcommand{\thefootnote}{\fnsymbol{footnote}}

\newcommand{\postscript}[2]
 {\setlength{\epsfxsize}{#2\hsize}
  \centerline{\epsfbox{#1}}}

\renewcommand{\baselinestretch}{1.0}

\setlength{\oddsidemargin}{-0.365in}
\setlength{\evensidemargin}{-0.365in}
\setlength{\topmargin}{-0.3in}
\setlength{\headheight}{0in}
\setlength{\headsep}{0in}
\setlength{\textheight}{10.1in}
\setlength{\textwidth}{7in}
\makeatletter \renewenvironment{proof}[1][Proof] {\par\pushQED{\qed}\normalfont\topsep6\p@\@plus6\p@\relax\trivlist\item[\hskip\labelsep\bfseries#1\@addpunct{.}]\ignorespaces}{\popQED\endtrivlist\@endpefalse} \makeatother
\makeatletter
\renewenvironment{solution}[1][Solution] {\par\pushQED{\qed}\normalfont\topsep6\p@\@plus6\p@\relax\trivlist\item[\hskip\labelsep\bfseries#1\@addpunct{.}]\ignorespaces}{\popQED\endtrivlist\@endpefalse} \makeatother
\title{VE477 HW1}
\author{Wu Jiayao 517370910257 }
\date{September 2019}

\begin{document}

\maketitle

\section{EX.1}
\subsection{}
	The probability is calculated as
	\begin{align*}
		P &= \frac{\binom{n}{k}1^k \cdot (n-1)^{n-k}}{n^k} \\
		& = (\frac{1}{n})^k(1-\frac{1}{n})^{n-k}\binom{n}{k}
	\end{align*}
\subsection{}
	The probability of exactly one slot having k hash keys is $nP_k$. If $k > \frac{1}{2}n$, there will only one slot that has k hash keys. So the probability $P_k^{'}=nP_k$. If $k \leq \frac{1}{2}n$, there may be more than one slots that have k hash keys. To ensure all of the slots with most hash keys have k hash keys, the probability $P_k^{'} \leq nP_k$.\\
	Overall, $P_k^{'} \leq nP_k$.
\subsection{}
	First is to prove that $\binom{n}{x}(\frac{k}{n})^x(1-\frac{k}{n})^{n-x} < \frac{k^x}{x!}e^{-k}$
	
	\begin{align*}
		f(x)&=\binom{n}{x}(\frac{k}{n})^x(1-\frac{k}{n})^{n-x}\\
		&=\frac{n(n-1)...(n-x+1)}{x!}\cdot\frac{k^x}{n^x}(1-\frac{k}{n})^{n-x}\\
		&=\frac{k^x}{x!}\cdot \frac{n(n-1)...(n-x+1)}{n^x} \cdot (1+\frac{1}{-\frac{n}{k}})^{(-\frac{n}{k})(-k+\frac{xk}{n})}\\
		&< \frac{k^x}{x!} \cdot 1 \cdot e^{-k}
	\end{align*}
	For this problem,
	\begin{align*}
		P_k & = (\frac{1}{n})^k(1-\frac{1}{n})^{n-k}\binom{n}{k} \\
			&<\frac{e^{-1}}{k!}
	\end{align*}
	According to Stirling formula,
	$$
		n! \approx \sqrt{2\pi n}(\frac{n}{e})^n
	$$
	which means,
	\begin{align*}
		P_k <\frac{e^{-1}}{k!} &\approx \frac{e^{-1}}{\sqrt{2\pi k}(\frac{k}{e})^k}\\
		&= \frac{e^k}{k^k} \cdot \frac{e^{-1}}{\sqrt{2\pi k}}\\
		&< \frac{e^k}{k^k}
	\end{align*}
\subsection{}
	Skipped
\subsection{}
	\begin{align*}
		E[M] &= \sum_{i=0}^n i\cdot P[M=i] \\
			& \leq (k-1)\sum_{i=1}^{k-1} P_i + n \sum_{i=k}^n P_i 
	\end{align*}
	Select a k that $k > \frac{c \log{n}}{\log{\log{n}}}$ and $k-1 \leq \frac{c \log{n}}{\log{\log{n}}}$. We can conclude that
	\begin{align*}
		E[M] & \leq (k-1)\sum_{i=1}^{k-1} P_i + n \sum_{i=k}^n P_i  \\
			& \leq \frac{c \log{n}}{\log{\log{n}}} P(M \leq \frac{c \log{n}}{\log{\log{n}}}) + n P(M>\frac{c \log{n}}{\log{\log{n}}})\\
	\end{align*}
	According to the conclusion in 1.4, $P(M>\frac{c \log{n}}{\log{\log{n}}}) < \frac{1}{n^2}$, therefore
	\begin{align*}
	E[M] &\leq \frac{c \log{n}}{\log{\log{n}}} P(M \leq \frac{c \log{n}}{\log{\log{n}}}) + n \sum_{i=k}^n \frac{1}{n^2} \\
	& \leq \frac{c \log{n}}{\log{\log{n}}}\cdot 1+ 1
	\end{align*}
	As a conclusion,
	$$
		E[M] = \mathcal{O}(\frac{\log{n}}{\log{\log{n}}})
	$$
\section{EX.2}
\begin{algorithm}[H]
	\BlankLine
	\SetKwInOut{Input}{input}
	\SetKwInOut{Output}{output}
	\caption{Determine the minimum spanning tree}\label{mst}
	\Input{the decreased edge E(u,v), minimum spanning tree T of graph G}
	\Output{new minimum spanning tree T'}
	\BlankLine
	$T' \gets T+E(u,v)$\\
	$L_0 \gets \text{The only cycle in T' that is formed due to }E(u,v)$\\
	$w_0 \gets \text{weight of }E(u,v)$\\
	$E' \gets \text{The edge in } L_0 \text{ that has the largest weight}$\\
	$T' \gets T'-E'$\\
	$\text{return } T'$
\end{algorithm}

\section{EX.3}
\subsection{}
    Skipped.
\subsection{}
	\subsubsection{}
    \begin{algorithm}[H]
        \BlankLine
        \SetKwInOut{Input}{input}
        \SetKwInOut{Output}{output}
    	\caption{mult(x,y)}\label{mult}
    	\Input{two integers to multiply x,y}
    	\Output{result of multiply n}
    	\BlankLine
    	\If{x=0 \textbf{or} y=0}{
    	    \textbf{return 0}
    	}
    	return $mult(2x,\lfloor y/2 \rfloor)+x \times (y \ mod \ 2)$
	\end{algorithm}
\newpage
	\subsubsection{}
		\par It is to prove that when mult(x,y) is right for integers $(2m,n)$, which means $mult(2m,n)=2mn$, then mult(x,y) is also correct for $(m,2n)$,$(m,2n+1)$.
		\par First, whether $m=0$ or $n=0$, the statement is proved true.
		\par Then
		$$
			mult(m,2n) = mult(2m,n)+ 2m*0 = 2mn=m \times 2n
		$$
		$$
			mult(m,2n+1) = mult(2m,n)+ 2m*1 = 2mn+2m = m \times (2n+1)
		$$
		\par Hence the statement is proved true. The correctness of the algorithm is proved.
\section{EX.4}
    \par The minimum number is 7.
    \par Divide the 25 horses into 5 groups of 5. Each group hold one race. Record the first, second, third fastest, noted as $h_{11},h_{12},h_{13}$ for the first, second, third of group 1, for example. Current race number is 5.
    \par Hold a race between $h_{11},h_{21},h_{31},h_{41},h_{51}$, record the first, second,third as $h_{61},h_{62},h_{63}$. $h_{61}$ is the fastest. Current race number is 6.
    \par Suppose that $h_{62}$ is $h_{i1}$ and  $h_{63}$ is $h_{j1}$. Hold a race between $h_{62}$,$h_{63}$,$h_{i2}$,$h_{i3}$,$h_{j2}$. Record the first, second, third as $h_{71},h_{72},h_{73}$. $h_{71}$ is the second fastest. $h_{72}$ is the third fastest.Current number is 7.
    
    
\section{EX.5}
\subsection{}
    \par Neither will solve the problem. A counterexample is to get $n=8$ from $S=\{2,3,5,7,9\}$ where the solution is  $\{3,5\}$
\subsection{}
    \par Skipped.
\subsection{}
    \par Problem: Use at least as possible coins to give out a combination of \$51. You have coins with value \$40, \$25, \$7, \$2, \$1.
    \par According to greedy algorithm, locally optimal is given by \$40,\$7,\$2,\$2, four coins.
    \par But actually the globally optimal is given by \$25,\$25,\$1, three coins.
\end{document}
