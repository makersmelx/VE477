\documentclass[12pt,a4paper]{article}
%\usepackage{ctex}
\usepackage{amsmath,amscd,amsbsy,amssymb,latexsym,url,bm,amsthm}
\usepackage{epsfig,graphicx,subfigure}
\usepackage{enumitem,balance}
\usepackage{wrapfig}
\usepackage{mathrsfs,euscript}
\usepackage[usenames]{xcolor}
\usepackage{hyperref}
\usepackage[vlined,ruled,commentsnumbered,linesnumbered]{algorithm2e}

\newtheorem{theorem}{Theorem}
\newtheorem{lemma}[theorem]{Lemma}
\newtheorem{proposition}[theorem]{Proposition}
\newtheorem{corollary}[theorem]{Corollary}
\newtheorem{exercise}{Exercise}
\newtheorem*{solution}{Solution}
\newtheorem{definition}{Definition}
\theoremstyle{definition}


%\numberwithin{equation}{section}
%\numberwithin{figure}{section}

\renewcommand{\thefootnote}{\fnsymbol{footnote}}

\newcommand{\postscript}[2]
 {\setlength{\epsfxsize}{#2\hsize}
  \centerline{\epsfbox{#1}}}

\renewcommand{\baselinestretch}{1.0}

\setlength{\oddsidemargin}{-0.365in}
\setlength{\evensidemargin}{-0.365in}
\setlength{\topmargin}{-0.3in}
\setlength{\headheight}{0in}
\setlength{\headsep}{0in}
\setlength{\textheight}{10.1in}
\setlength{\textwidth}{7in}
\makeatletter \renewenvironment{proof}[1][Proof] {\par\pushQED{\qed}\normalfont\topsep6\p@\@plus6\p@\relax\trivlist\item[\hskip\labelsep\bfseries#1\@addpunct{.}]\ignorespaces}{\popQED\endtrivlist\@endpefalse} \makeatother
\makeatletter
\renewenvironment{solution}[1][Solution] {\par\pushQED{\qed}\normalfont\topsep6\p@\@plus6\p@\relax\trivlist\item[\hskip\labelsep\bfseries#1\@addpunct{.}]\ignorespaces}{\popQED\endtrivlist\@endpefalse} \makeatother
\title{VE477 HW1}
\author{Wu Jiayao 517370910257 }
\date{September 2019}

\begin{document}

\maketitle

\section{EX.1}
\subsection{}
\section{EX.2}
\begin{algorithm}[H]
	\BlankLine
	\SetKwInOut{Input}{input}
	\SetKwInOut{Output}{output}
	\caption{Determine the minimum spanning tree}\label{mst}
	\Input{the decreased edge (u,v), minimum spanning tree T of graph G}
	\Output{new minimum spanning tree T'}
	\BlankLine
	$T' \gets T$\;
	\For{$\text{vertex x in G that is connected with u or v by edges}$}{
	    \If{$edge(x,u) \in T\ \textbf{and} \ edge(x,v) \in T$}{
	        \If{$w(u,v) \textless \text{the maximum of w(x,u) and w(x,v)}$}{
	            Delete from T' the one between the two edges edge(x,u),edge(x,v) with the larger weight  \\
	            Add edge(u,v) to T'
	       }
	    }
	}
	
	
\end{algorithm}

\section{EX.3}
\subsection{}
    Skipped.
\subsection{}
    \begin{algorithm}[H]
        \BlankLine
        \SetKwInOut{Input}{input}
        \SetKwInOut{Output}{output}
    	\caption{Mult(x,y)}\label{mult}
    	\Input{two numbers to multiply x,y}
    	\Output{result n}
    	\BlankLine
    	\If{x=0 \textbf{or} y=0}{
    	    \textbf{return 0}
    	}
    	return $Mult(2x,\lfloor y/2 \rfloor)+x \times (y \ mod \ 2)$
    \end{algorithm}
\newpage
\section{EX.4}
    \par The minimum number is 8.
    \par Divide the 25 horses into 5 groups of 5. Each group hold one race. Record the first, second, third fastest, noted as $h_{11},h_{12},h_{13}$ for the first, second, third of group 1, for example. Current race number is 5.
    \par Hold a race between $h_{11},h_{21},h_{31},h_{41},h_{51}$, record the first, second,third as $h_{61},h_{62},h_{63}$. $h_{11}$ is the fastest. Current race number is 6.
    \par Suppose that $h_{63}$ is $h_{n1}$. Hold a race between $h_{62}$ and all of the $h_{x2}(x\in[1,5],x\neq n)$. Record the first, second, third as $h_{71},h_{72},h_{73}$. $h_{71}$ is the second fastest. Current number is 7.
    \par \par Suppose that $h_{72}$ is $h_{i2}$, $h_{73}$ is $h_{j2}$. Hold a race between all of the $h_{x3}(x\in[1,7],x\neq i,j)$. The first in this race is the third fastest. Current number is 8.
    
\section{EX.5}
\subsection{}
    \par Fit the knapsack with the largest items first solves the problem.
\subsection{}
    \par Skipped.
\subsection{}
    \par Problem: Use at least as possible coins to give out a combination of \$51. You have coins with value \$40, \$25, \$7, \$2, \$1.
    \par According to greedy algorithm, locally optimal is given by \$40,\$7,\$2,\$2, four coins.
    \par But actually the globally optimal is given by \$25,\$25,\$1, three coins.
\end{document}
