\documentclass[12pt,a4paper]{article}
%\usepackage{ctex}
\PassOptionsToPackage{unicode}{hyperref}
\PassOptionsToPackage{hyphens}{url}
\usepackage{amsmath,amscd,amsbsy,amssymb,latexsym,url,bm,amsthm}
\usepackage{epsfig,graphicx,subfigure}
\usepackage{enumitem,balance}
\usepackage{wrapfig}
\usepackage{mathrsfs,euscript}
\usepackage[usenames]{xcolor}
\usepackage[colorlinks,linkcolor = blue]{hyperref}
\usepackage[vlined,ruled,commentsnumbered,linesnumbered]{algorithm2e}
\usepackage{listings}
\usepackage[utf8]{inputenc}
\usepackage{algorithmicx}
\usepackage{algpseudocode}
\usepackage{amsmath}

\newtheorem{theorem}{Theorem}
\newtheorem{lemma}[theorem]{Lemma}
\newtheorem{proposition}[theorem]{Proposition}
\newtheorem{corollary}[theorem]{Corollary}
\newtheorem{exercise}{Exercise}
\newtheorem*{solution}{Solution}
\newtheorem{definition}{Definition}
\theoremstyle{definition}


%\numberwithin{equation}{section}
%\numberwithin{figure}{section}

\renewcommand{\thefootnote}{\fnsymbol{footnote}}

\newcommand{\postscript}[2]
 {\setlength{\epsfxsize}{#2\hsize}
  \centerline{\epsfbox{#1}}}

\renewcommand{\baselinestretch}{1.0}

\setlength{\oddsidemargin}{-0.365in}
\setlength{\evensidemargin}{-0.365in}
\setlength{\topmargin}{-0.3in}
\setlength{\headheight}{0in}
\setlength{\headsep}{0in}
\setlength{\textheight}{10.1in}
\setlength{\textwidth}{7in}
\makeatletter \renewenvironment{proof}[1][Proof] {\par\pushQED{\qed}\normalfont\topsep6\p@\@plus6\p@\relax\trivlist\item[\hskip\labelsep\bfseries#1\@addpunct{.}]\ignorespaces}{\popQED\endtrivlist\@endpefalse} \makeatother
\makeatletter
\renewenvironment{solution}[1][Solution] {\par\pushQED{\qed}\normalfont\topsep6\p@\@plus6\p@\relax\trivlist\item[\hskip\labelsep\bfseries#1\@addpunct{.}]\ignorespaces}{\popQED\endtrivlist\@endpefalse} \makeatother

\SetKwInOut{Input}{Input}
\SetKwInOut{Output}{Output}
\SetKwProg{Fn}{Function}{\string:}{end}

\usepackage{color}
\definecolor{codegreen}{rgb}{0,0.6,0}
\definecolor{codegray}{rgb}{0.5,0.5,0.5}
\definecolor{codepurple}{rgb}{0.58,0,0.82}
\definecolor{backcolour}{rgb}{0.95,0.95,0.92}
 
\lstdefinestyle{mystyle}{  
    commentstyle=\color{codegreen},
    keywordstyle=\color{blue},
    numberstyle=\tiny\color{codegray},
    stringstyle=\color{codepurple},
    basicstyle=\footnotesize,
    breakatwhitespace=false,         
    breaklines=true,                 
    captionpos=b,                    
    keepspaces=true,                 
    numbers=left,                    
    numbersep=5pt,                  
    showspaces=false,                
    showstringspaces=false,
    showtabs=false,                  
    tabsize=2,
    frame=shadowbox
}
\lstset{style=mystyle}

\title{VE477 HW 5}
\author{Wu Jiayao 517370910257}

\begin{document}
\maketitle
\section{The partition problem}
\subsection{Explanation}
Omit
\subsection{A simple strategy}
It is not a good solution. A counter example is partition {1,2}, {3,4} for {1,2,3,4}. It doesn't solve the partition problem.
\subsection{Recursive Algorithm}
Denote $s_i$ the size of $i^{th}$ group in the partition. Transform the problem into find the minium value of the larger one between the sum of the $k^{th}$ group during the partition and $M(n-s_k,k-1)$ for the $k-1$ group before. Then recursively, $M(n-s_k,k-1)$ is to calculated between the $k-1^{th}$ group and $k-2$ groups before.
\subsection{Complexity}
There is a total of $k \times n$ combinations. Trasversing takes at most $n^2$. Therefore,
$$
T(n) = \mathcal{O}(N^3)
$$
\subsection{Stored quantites}
Denote $s[i]$ the element in the set. $p[i]=\sum_{k=1}^{i}s_k$ could be stored. 
\subsection{Dynamic Programming}
\begin{algorithm}[H]
    \Input{A given arrangement S of non-negative numbers $s_1,s_2, \cdots,s_n$ and an integer k.}
    \Output{Partition S into k ranges, so as to minimize the maximum sum over all the ranges}
    $p = []$ \\
    $p[0]\ =\ 0$ \\
    \For{$i\ =\ 1\ to\ n$}{
        $p[i]\ =\ s_i+p[i-1]$ \\
    }
    \For{$i\ =\ 1\ to\ n$}{
        $M[i,1]\ =\ p[i]$ \\
    }
    \For{$j\ =\ 1\ to\ k$}{
        $M[1,j]\ =\ s_1$ \\
    }
    \For{$i\ =\ 2\ to\ n-1$}{
        \For{$j\ = 2\ to\ k$}{
            $M[i,j]=\infty$ \\
            \For{$x\ =\ 1\ to\ i-1$}{
                $s\ = max(M[x,j-1],p[i]-p[x])$ \\
                \If{$M[i,j]>s$}{
                    $M[i,j]=s$ \\
                    $D[i,j]=x$
                }
            }
        }
    }
    \Return $M[n,k]$ 
\end{algorithm}
\subsection{Proof of correctness}
For $n=1$ or $k=1$, all the value in matrix $M$ is correct. If $M[x,j-1]$ is correct, $M[x,j]$ is sure to be correct, and will not affect the value of $M[x,j-1]$. Hence, the algorithm is correct.
\subsection{Time complexity}
As written above, there are at most $k \times n \times n$ times of operations. The time complexity is $\mathcal{O}(kn^2)$.
\subsection{Where to set the dividers}
As already implemented above, the last position of partition is $D[n,k]$. Then combine with $M[n,k]$ to search from the back to find the value of the $k-1^{th},k-2^{th},\cdots,1^{st}$ partition.
\section{Critical Thinking}
We can get a generator which generates the range $0-24$ by $5 \times B()+B()$. Similarly, the range $0-624$ can be generated. Therefore, the range $0-5^{2^n},n \in \mathcal{Z^+}$ can be generated.\\
\begin{algorithm}[H]
    \Input{B()}
    \Output{A random number in the range $0-n$}
    $i$ satisfies $5^{2^{i-1}}\ <\ n+1\ <=\ 5^{2^i}$ \\
    Let $B_i()$ be the related generator. \\
    $num = \lfloor 5^{2^i}/(n+1) \rfloor \cdot (n+1)$ \\
    \While{$G\ =\ B_i()\ >\ num$}{
        continue
    }
    \Return $G\ \%\ (n+1)$
\end{algorithm}
\section{Bellman-Ford algorithm}
\begin{algorithm}[H]
    \Input{A directed weighted graph $G=\langle V, E\rangle$}
    \Output{Whether there exists negative cycles}
    Randomly choose a vertex $s$. \\
    \ForEach{$v\ \in \ V$}{
        \If{$v$ is $s$}{
            $dist[v]\ =\ 0$
        }
        \Else{
            $dist[v]\ =\ \infty$
        }
    }
    \For{$i\ =\ 2\ to\ | V |$}{
        \ForEach{$(u,v) \in E$}{
            $tmp\ =\ dist[u]\ +\ weight(u,v)$ \\
            \If{$tmp\ <\ dist[v]$}{
                $dist[v]\ =\ tmp$
            }
        }
    }
    \For{$i\ =\ 2\ to\ | V |$}{
        \ForEach{$(u,v) \in E$}{
            $tmp\ =\ dist[u]\ +\ weight(u,v)$ \\
            \If{$tmp\ <\ dist[v]$}{
                \Return True
            }
        }
    }
    \Return False
\end{algorithm}
\section{Augmenting path}
Omitted.
\section{Wifi Problem}
\begin{algorithm}[H]
    \caption{$WIFI(r,l,loc,pos)$} 
    \Input{$r,\ l\ ,locations\ of\ k\ hostpots,\ positions\ of\ n\ users$}
    \Output{Whether all the users can connect to Wifi}
    Create an empty graph $G \langle V, E \rangle$ \\
    $V$ $\gets$ $V\ +\ \{s,t\}$ \\
    \For{$i\ in\ range(0,n)$}{
        \For{$j\ in\ range(0,k)$}{
            \If{Distance(user $i$, hostpot $j$) $<$ $r$}{
                Add the edge $(u[i],h[j])$to $E$ with capacity of 1
            }
        }
    }
    \For{$i\ in\ range(0,n)$}{
        Add the edge $(s,u[i])$ to $E$ with capacity of 1 
    }
    \For{$i\ in\ range(0,k)$}{
        Add the edge $(h[i],t)$ to $E$ with capacity of $l$
    }
    $F$ $\gets$ $Edmonds-Karp(G)$ \\
    \Return $F\ ==\ n$
\end{algorithm}
The time complexity is the same as Edmonds-Karp algorithm, which is $\mathcal{O}(|V||E|^2)$. 
Capacity of $(s, u[i])$ and $(u[i],h[j])$ is 1 means that one user can only access to one hostpot. Capacity of $(u[i],h[j])$ is $l$ restircts the max number of users . The graph is constructed that the condition of max flow happens when all the users are connected to Wifi. Hence, it is correct.
% \section*{Reference}
% {\noindent [1] \url{cs.nthu.edu.tw/~wkhon/toc07-lectures/lecture21.pdf}}
\end{document}