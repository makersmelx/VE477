\documentclass[12pt,a4paper]{article}
%\usepackage{ctex}
\PassOptionsToPackage{unicode}{hyperref}
\PassOptionsToPackage{hyphens}{url}
\usepackage{amsmath,amscd,amsbsy,amssymb,latexsym,url,bm,amsthm}
\usepackage{epsfig,graphicx,subfigure}
\usepackage{enumitem,balance}
\usepackage{wrapfig}
\usepackage{mathrsfs,euscript}
\usepackage[usenames]{xcolor}
\usepackage[colorlinks,linkcolor = blue]{hyperref}
\usepackage[vlined,ruled,commentsnumbered,linesnumbered]{algorithm2e}
\usepackage{listings}
\usepackage[utf8]{inputenc}
\usepackage{algorithmicx}
\usepackage{algpseudocode}
\usepackage{amsmath}

\newtheorem{theorem}{Theorem}
\newtheorem{lemma}[theorem]{Lemma}
\newtheorem{proposition}[theorem]{Proposition}
\newtheorem{corollary}[theorem]{Corollary}
\newtheorem{exercise}{Exercise}
\newtheorem*{solution}{Solution}
\newtheorem{definition}{Definition}
\theoremstyle{definition}


%\numberwithin{equation}{section}
%\numberwithin{figure}{section}

\renewcommand{\thefootnote}{\fnsymbol{footnote}}

\newcommand{\postscript}[2]
 {\setlength{\epsfxsize}{#2\hsize}
  \centerline{\epsfbox{#1}}}

\renewcommand{\baselinestretch}{1.0}

\setlength{\oddsidemargin}{-0.365in}
\setlength{\evensidemargin}{-0.365in}
\setlength{\topmargin}{-0.3in}
\setlength{\headheight}{0in}
\setlength{\headsep}{0in}
\setlength{\textheight}{10.1in}
\setlength{\textwidth}{7in}
\makeatletter \renewenvironment{proof}[1][Proof] {\par\pushQED{\qed}\normalfont\topsep6\p@\@plus6\p@\relax\trivlist\item[\hskip\labelsep\bfseries#1\@addpunct{.}]\ignorespaces}{\popQED\endtrivlist\@endpefalse} \makeatother
\makeatletter
\renewenvironment{solution}[1][Solution] {\par\pushQED{\qed}\normalfont\topsep6\p@\@plus6\p@\relax\trivlist\item[\hskip\labelsep\bfseries#1\@addpunct{.}]\ignorespaces}{\popQED\endtrivlist\@endpefalse} \makeatother

\SetKwInOut{Input}{Input}
\SetKwInOut{Output}{Output}
\SetKwProg{Fn}{Function}{\string:}{end}

\usepackage{color}
\definecolor{codegreen}{rgb}{0,0.6,0}
\definecolor{codegray}{rgb}{0.5,0.5,0.5}
\definecolor{codepurple}{rgb}{0.58,0,0.82}
\definecolor{backcolour}{rgb}{0.95,0.95,0.92}
 
\lstdefinestyle{mystyle}{  
    commentstyle=\color{codegreen},
    keywordstyle=\color{blue},
    numberstyle=\tiny\color{codegray},
    stringstyle=\color{codepurple},
    basicstyle=\footnotesize,
    breakatwhitespace=false,         
    breaklines=true,                 
    captionpos=b,                    
    keepspaces=true,                 
    numbers=left,                    
    numbersep=5pt,                  
    showspaces=false,                
    showstringspaces=false,
    showtabs=false,                  
    tabsize=2,
    frame=shadowbox
}
\lstset{style=mystyle}

\title{VE477 HW 4}
\author{Wu Jiayao 517370910257}

\begin{document}
\maketitle

\section{}
\subsection{}
$$
T(2^{64})=\frac{2^{64}}{33.86 \times 10^{15}}=544.8 s 
$$

$$
T(2^{80})=\frac{2^{80}}{33.86 \times 10^{15}}=3.6 \times 10^{7}s
$$

\subsection{}
$$
Nums\left(2^{64}\right)=\left\lceil\frac{2^{64}}{3.8 \times 10^{9} \times 86400}\right\rceil= 56186
$$

$$
Nums\left(2^{80}\right)=\left\lceil\frac{2^{64}}{3.8 \times 10^{9} \times 86400 \times 31}\right\rceil= 1.19 \times 10^{8}
$$

\subsection{}
$$
Nums(2^{64}) = \frac{2^{64}}{8 \times 10^{12} \times 16}=1.4 \times 10^{5}
$$

$$
Nums(2^{80}) = \frac{2^{80}}{8 \times 10^{12} \times 16}=9.4 \times 10^{9}
$$

\section{}
\begin{enumerate}
    \item create an array of size k with the first k elements in S,named $S^{'}$.
    \item For the rest elements, each time when visiting an element $A[i]$, generate a random number $t = rand()\%n$, where $n$ is the size of S. If $t \leq k$, then $S^{'}[t]=A[i]$.
\end{enumerate}
\section{}
\subsection{}
\begin{algorithm}[H]
\Input{Number of layer i}
\Output{Sum of the ith layer}
\Return $3^{i-1}$
\end{algorithm}
\subsection{}
The complexity is $\mathcal{O}(i)$ for $3^{i-1}$.
\section{}
Omit.
\section{}
\subsection{}
Omit
\subsection{}
Given a graph and a clique of k vertices, it uses at worst $\mathcal{O}(n^2)$ time to check whether the points are adjacent to each
other. The answer can be worked out in polynomial time. Thus it is in $\mathcal{NP}$.
\subsection{}
Let $C_1,C_2,\cdots,C_k$ be the clauses in F. Let $x_{j,1}, x_{j,2}, x_{j,3}$ be the literals of Cj. 
\begin{enumerate}
    \item For each literal $x_{j,q}$, create a distinct vertex in G representing it.
    \item Remove edges that join two vertices which are in the same clause.
    \item Remove edges that join two vertices whose literals
    is the negation of the others
\end{enumerate}
\subsection{}
It is $\mathcal{NP}-complete$.
\section{}
\subsection{}
Omit.
\subsection{}
Given an undirected graph $G$ and an integer k, determine whether $G$ has a independent subset of size k.
\subsection{}
It takes $\mathcal{O}(\left| V \right|)$ time in worst cases to check all the vertices. Since this is in polynomial time, this problem is in $\mathcal{NP}$.
\subsection{}
Given $G=(V,E)$, we set $G’=(V’,E’)$ to be the complement of $G$ where $V=V^{'}$, $G^{'}$ is the graph we construct.
\subsection{}
It is $\mathcal{NP}$-complete problem.
\section*{Reference}
{\noindent [1] \url{cs.nthu.edu.tw/~wkhon/toc07-lectures/lecture21.pdf}}
\end{document}